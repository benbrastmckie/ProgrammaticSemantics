\documentclass[a4paper, 11pt]{article} % Font size (can be 10pt, 11pt or 12pt) and paper size (remove a4paper for US letter paper)

\usepackage[protrusion=true,expansion=true]{microtype} % Better typography
\usepackage{graphicx} % Required for including pictures
\usepackage{wrapfig} % Allows in-line images
\usepackage{enumitem} %%Enables control over enumerate and itemize environments
\usepackage{setspace}
\usepackage{amssymb, amsmath, mathrsfs,mathabx} %%Math packages
\usepackage{stmaryrd}
\usepackage{mathtools}
\usepackage{multicol} 
\usepackage{mathpazo} % Use the Palatino font
\usepackage[T1]{fontenc} % Required for accented characters
\usepackage{array}
\usepackage{bibentry}
\usepackage[dvipsnames]{xcolor}  % Color support

%% ========== Hyperlinks (load last) ==========

\usepackage[
    linktocpage=true,
    pdfusetitle
]{hyperref}

% Configure hyperlink colors
\definecolor{URLblue}{RGB}{0,0,150}
\hypersetup{
    colorlinks   = true,         % Use colored links instead of boxes
    urlcolor     = URLblue,      % Color for external hyperlinks
    linkcolor    = URLblue,      % Color of internal links
    citecolor    = black          % Color of citations
}

%%% CITATIONS %%%
% \usepackage{bibentry} %%Replace \bibliography{} with \nobibliography{} for no bib
\usepackage[round]{natbib} %%Or change 'round' to 'square' for square backers
\setcitestyle{aysep={}}
 %   \citet{key} ==>>                Jones et al. (1990)
 %   \citet*{key} ==>>               Jones, Baker, and Smith (1990)
 %   \citep{key} ==>>                (Jones et al., 1990)
 %   \citep*{key} ==>>               (Jones, Baker, and Smith, 1990)
 %   \citep[chap. 2]{key} ==>>       (Jones et al., 1990, chap. 2)
 %   \citep[e.g.][]{key} ==>>        (e.g. Jones et al., 1990)
 %   \citep[e.g.][p. 32]{key} ==>>   (e.g. Jones et al., p. 32)
 %   \citeauthor{key} ==>>           Jones et al.
 %   \citeauthor*{key} ==>>          Jones, Baker, and Smith
 %   \citeyear{key} ==>>             1990
\usepackage{etoolbox} %%For \citepos
\usepackage{xstring} %%For \citepos

\makeatletter %definition of \citepos
% \patchcmd{\NAT@test}{\else \NAT@nm}{\else \NAT@nmfmt{\NAT@nm}}{}{} %turn on for numeric citations
\DeclareRobustCommand\citepos% define \citepos
  {\begingroup
   \let\NAT@nmfmt\NAT@posfmt% same as for citet except with a different name format
   \NAT@swafalse\let\NAT@ctype\z@\NAT@partrue
   \@ifstar{\NAT@fulltrue\NAT@citetp}{\NAT@fullfalse\NAT@citetp}
  }
   
\let\NAT@orig@nmfmt\NAT@nmfmt %makes adapt to last names ending with an 's'.
\def\NAT@posfmt#1{%
  \StrRemoveBraces{#1}[\NAT@temp]%
  \IfEndWith{\NAT@temp}{s}
    {\NAT@orig@nmfmt{#1'}}
    {\NAT@orig@nmfmt{#1's}}}
\makeatother

\newcommand{\corner}[1]{\ulcorner#1\urcorner} %%Corner quotes
\newcommand{\tuple}[1]{\langle#1\rangle} %%Angle brackets
\newcommand{\set}[1]{\lbrace#1\rbrace} %%Set brackets
\newcommand{\abs}[1]{|#1|} %%Set brackets
\newcommand{\interpret}[1]{\llbracket#1\rrbracket} %%Double brackets
\newcommand{\N}{\mathbb{N}}
\newcommand{\D}{\mathbb{D}}
\newcommand{\Z}{\mathbb{Z}}
\newcommand{\Q}{\mathbb{Q}}
\newcommand{\R}{\mathbb{R}}

\makeatletter
\renewcommand\@biblabel[1]{\textbf{#1.}} % Change the square brackets for each bibliography item from '[1]' to '1.'
\renewcommand{\@listI}{\itemsep=0pt} % Reduce the space between items in the itemize and enumerate environments and the bibliography

\renewcommand{\maketitle}{ % Customize the title - do not edit title and author name here, see the TITLE block below
\begin{flushright} % Right align
{\LARGE\@title} % Increase the font size of the title

\vspace{10pt} % Some vertical space between the title and author name

{\@author} % Author name
\\\@date % Date

\vspace{30pt} % Some vertical space between the author block and abstract
\end{flushright}
}

%----------------------------------------------------------------------------------------
%	TITLE
%----------------------------------------------------------------------------------------

\title{\textbf{Programmatic Truthmaker Semantics}} % Subtitle

\author{\textsc{Advances in Truthmaker Semantics: II}\\ % Topic
  \em Benjamin Brast-McKie and  % Author
  Miguel Buitrago  % Author
} 

\date{\today} % Date

%----------------------------------------------------------------------------------------

\begin{document}

\maketitle % Print the title section

\thispagestyle{empty}

%----------------------------------------------------------------------------------------

\nocite{Brast-McKieforthcoming}

\vspace{-50pt}

\section*{\it Abstract}

I demonstrate in Brast-McKie (\href{https://link.springer.com/article/10.1007/s10992-025-09793-8}{2025}) how to define \citepos{Fine2012a,Fine2012b} imposition relation in terms of the primitives that \cite{Fine2017d,Fine2017,Fine2017a} includes in a modalized state space to provide a logic for counterfactual conditionals which is at least strong.
The second half of the paper extends the resulting truthmaker semantics to accommodate tense operators by introducing a primitive \textit{task relation} in terms of which the possible states, world states, and world histories may then be defined.

This talk presents these additions to Fine's truthmaker semantics as case studies of a \textit{programmatic methodology} which makes use of the \href{https://pypi.org/project/model-checker/}{\texttt{model-checker}} software that I developed with Miguel Buitrago alongside this project.\footnote{\texttt{https://pypi.org/project/model-checker/}}
After presenting the semantics using traditional model theory in the first part of the talk, I will demonstrate the utility of implementing a semantics using the model-checker in order to find finite countermodels and establish logical consequences up to a user defined limit 2$^\texttt{N}$ on the size of models.

By ruling out finite countermodels smaller than 2$^\texttt{N}$, the \texttt{model-checker} provides evidence that a logical consequence under consideration has no countermodels.
Rather than a substitute for working in traditional model theory, implementing a programmatic semantics with the \texttt{model-checker} greatly eases the process of prototyping new semantic theories and making novel additions to existing theories.
In addition to allowing user to rapidly explore the extension of the semantic theories that they implement, the \texttt{model-checker} provides resources for easily uploading semantic systems to the \texttt{TheoryLib}, providing access for other users to explore.

Implementations are also modular and easy to combine and compare, allowing users to survey the interactions in languages with many different operators at no extra computational cost.
Moreover, the \textit{computability} of a semantics provides an objective measure on the complexity of a theory which may be weighed alongside other theoretical virtues.

Although the \texttt{model-checker} is a general purpose utility for working in semantics, applications in truthmaker semantics are particularly natural given the increased complexity of these semantic systems. 
Rather than a deficiency, I will characterize well-motivated forms of theoretical complexity as a sign of the maturity of semantics as a discipline.
It is in support of both the future development and accessibility of truthmaker semantics that the \texttt{model-checker} aims to make a contribution.






\vfill

\bibliographystyle{Phil_Review} %%bib style found in bst folder, in bibtex folder, in texmf folder.
\bibliography{Zotero} %%bib database found in bib folder, in bibtex folder


\end{document}
